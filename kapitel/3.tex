(här bör vi ha en inledning om vi vill dela upp detta i kapitel)


\subsection{Kraftförenkling}
\textbf{(kommer aldrig ihåg att detta var ett problem...). Jo, fast var inte förenklingar av krafter. Det var bara en banal del i framtagandet av modellen som vi inte gjorde korrekt från början.}

För att kunna skapa ett system måste man förenkla alla krafter och modellera dem tillsammans. Vi stötte på problem då vi inte fick vår modellerade lina att sluta oscillera. Vi ville inte använda oss av luftmotstånd som vår dämpningskraft då vi ville hålla vårt system simpelt. Vi prövade sedan att dela upp friktionskraften i sina komposanter, detta fungerade tillslut och modellerade systemet bra. Vad man kan lära sig av detta är att det är viktigt att förenkla systemet men ju mer man förenklar desto mer information förloras. Det är viktigt att hitta en medelpunkt där man tar med så få faktorer som möjligt men fortfarande får ut den infomationen man behöver.

\subsection{Uteslutning av rampsvar}


\subsection{Dämpningskontantens enhet}
\textbf{(Detta är ett litet problem som vi löste, det är inget som ses när läsaren går igenom rapporten och undrar över och därför behövs detta inte diskuteras)}

I kapitel två skulle vi välja passande systemparametrar till vårt system. Bland dessa fanns dämpningkonstanten inkluderad. Då dess enhet inte direkt är uppenbar spekulerade vi att den kunde vara enhetslös. Då detta kändes underligt senare i rapporten valde vi att genomföra en enhetsanalys som producerade enheten kilogram per sekund. Detta kändes underligt men även rimligt om man såg det som att dämpning uttrycks i mängden kilogram som hålls tillbaka per sekund.

\subsection{Variation av konstanter för amplitudkaraktäristiken}
När vi redogjorde för vårt systems amplitudkaraktäristik i kapitel 2 valde vi även att undersöka vad som händer om vi varierar dämpningskonstanten. På samma sätt hade vi kunnat variera massan och fjäderkonstanten för att visa hur amplitudkaraktäristiken varierar då. Vi valde att inte göra detta då vi såg att detta inte gav några särdeles intressanta resultat.

\subsection{Jämförelse med verkligheten}
\textbf{Vi kan dock få en känsla om våra resultat är rimliga eller inte...}

Det skulle vara intressant att ta reda på hur väl våra presenterade resultat överensstämmer med verkligheten. Detta då det skulle ge oss en bra bild över hur väl vårt system representerar det verkliga scenariot som vi försöker modellera. Tyvärr har det inte varit möjligt att omfatta det i denna rapport då vi inte har haft tillgång till de resurserna som skulle krävas för det.
