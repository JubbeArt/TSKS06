För att skapa en hanterlig modell krävdes vissa förenklingar. En sådan förenkling var att vi endast betraktade längsriktningen och höjdriktning. I verkligenheten är svängningar i sidriktning betydligt svårare att hantera för en lindansare, men uppstår i något mindre grad än svängningar i höjdriktningen. Några andra förenklingar var att vi såg personen som en punktmassa med endast en kontaktpunkt med linan och vi bortsåg från vind, temperatur och linans egna vikt. Betraktar man videoklipp av riktiga lindansare tycks vår modell stämma bra överens med verkligheten trots alla förenklingar. För att faktiskt testa hur väl vår modell stämmer överens skulle vi behöva experimentera med en verklig lina och lindansare. Vi hade dock inte resursers nog att genomföra detta. 

Då rampsvar inte kändes relevant för vårt system valdes analysen av denna insignal bort. Att lindansaren påverkas av en linjärt växande kraft är inte en realistisk situation. Till skillnad från impulssvaret ger inte rampsvaret någon vidare förståelse för systemets fundamentala egenskaper och är inte nödvändigt för beräknandet av andra systembeskrivande funktioner.