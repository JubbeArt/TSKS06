I förra kapitlet togs en differentialekvation fram som beskriver vårt system som en relation mellan insignalen och utsignalen. Utifrån denna differentialekvation kan nu olika systemegenskaper tas fram som gör det möjligt att analysera systemet. Vi kommer få olika system med olika egenskaper beroende på vilka  systemparametrar som väljs. Systemparametrarna i vårt fall är massan $m$, fjäderkonstanten $k$ och dämpningskonstanten $c$. Exempelvis kommer linan svänga olika beroende på personens vikt. 

Systemanalysen kommer titta närmare på följande funktioner: systemfunktionen, impulssvaret, stegsvaret, frekvensfunktionen samt amplitud- och faskaraktäristiken. Vissa av dessa är funktioner som existerar i tidsdomänen och andra i frekvensdomänen. Eftersom det är grundläggande att förstå hur dessa domäner förhåller sig till varandra kommer detta diskuteras kort i nästa kapitel.

\newpage
\subsection{Frekvensdomänen}
Tidigare i rapporten har vi betraktat insignalen och utsignalen som funktioner av tid. Vi kommer nu även betrakta hur signalerna beter sig i frekvensdomänen där man istället undersöker vilka frekvenser en signal är uppbyggd av.

Exempelvis kan signalen $x(t)=3sin(2t) + sin(4t)$ ses som summan av två amplituder i tidsdomänen. I frekvensdomänen skulle den dock vara uppdelad i dess frekvenskomponenter, i detta fall två komponenter vid frekvenserna $2$ respektive $4$ rad/s. Detta visas grafiskt i figuren nedan.

\begin{figure}[h] 
    \centering
    \includegraphics{bilder/tid_vs_frekvens_exempel}
    \caption{Tidsdomänen och frekvensdomänen av signalen $x(t)=3sin(2t)+sin(4t)$}
    \label{fig:tid_vs_frekvens_exempel}
\end{figure}

För systemanalysen kommer vi använda oss av transformer som överför funktioner från tidsdomänen till frekvensdomänen, nämligen Laplacetransformen och Fouriertransformen.
Laplacetransformen är ett vanligt verktyg för att lösa differentialekvationer. För att ta sig tillbaka till tidsdomänen från frekvensdomänen kan inverstransformer appliceras.

\subsection{Systemfunktion}
Systemfunktionen $H(s)$ beskriver förhållandet mellan utsignalen och insignalen i frekvensdomänen. Detta innebär att det går att räkna ut en utsignal om man vet insignalen i frekvensdomänen och systemfunktionen. 
Det går även att utföra detta i tidsdomänen med faltning men det är oftast enklare att utföra beräkningen i frekvensdomänen.

För att beräkna systemfunktionen kommer hela differentialekvationen för systemet att Laplacetransformeras till frekvensdomänen. Det finns då två versioner av Laplacetransformen som kan användas: den enkelsidiga och den dubbelsidiga.
Eftersom vårt system är kausalt, det vill säga att utsignalen inte beror på på framtida insignaler, kommer den enkelsidiga transformen att användas.
Den enkelsidiga Laplacetransformen för en funktion $x(t)$ definieras som:
$$X(s) = \mathcal{L}\big\{x(t)\big\} = \int\limits_{0-}^{\infty} x(t)e^{-st}\,dt$$
där $s$ är en komplexvärd frekvens vanligtvis betecknad $s=\sigma+j\omega$.
Här används versaler för att beteckna funktioner i frekvensdomänen.
Denna transform behöver inte vara definierad för alla $s$ utan kan divergera i vissa fall. Därför är det viktigt att ange var den transformerade funktionen är definierad. 

Eftersom dessa integralberäkningar kan bli både långa och krångliga kommer vi i denna rapport använda tabellerna i häftet \textit{Formler \& Tabeller}\footnote{Sune Söderkvist, \textit{Formler \& Tabeller}, 4:e upplagan (2007)} för att beräkna Laplacetransformerna.
För att transformera differentialekvationen kommer följande samband för derivering i tidsdomänen att användas enligt tabell 18.7:
$$\mathcal{L}\bigg\{\frac{dy(t)}{dt}\bigg\} = sY(s)-y(0-)$$
Då vårt system är energifritt kommer alla $y(0-)$-termer vara lika med noll. Vår differentialekvation kan alltså Laplacetransformeras enligt:
$$ \mathcal{L}\bigg\{m\displaystyle\frac{d^2y(t)}{dt^2} + c\displaystyle\frac{dy(t)}{dt} + ky(t)\bigg\}= \mathcal{L}\bigg\{x(t)\bigg\} $$
\begin{center}$ \Longleftrightarrow \bigg/$ Tabell $18.7$, $18.8\,\bigg/$ \end{center}
$$ \Longleftrightarrow\, ms^2Y(s)+csY(s)+kY(s)=X(s)$$

Från ekvationen ovan kan nu systemfunktionen $H(s)$ bestämmas då den definieras som kvoten mellan utsignalen och insignalen i frekvensdomänen. Genom att bryta ut termen $Y(s)$ i vänsterledet och sedan dela med $X(s)$ och $s$-polynomet i båda led ges följande: 
$$H(s)=\frac{Y(s)}{X(s)}=\frac{1}{ms^2+cs+k}$$
\newline

\subsubsection{Pol-nollställediagram}
Utifrån systemfunktionen kan nu flera egenskaper om systemet tas fram. Oftast är det intressant att studera nollställena för polynomen i systemfunktionen. Täljarpolynomets rötter kallas systemfunktionens nollställen medan nämnarpolynomets rötter kallas systemfunktionens poler. Man kan direkt se att systemfunktionen inte har några nollställen då täljarpolynomet inte har några rötter. För att hitta polerna måste först rötterna till nämnarpolynomet hittas, detta kan till exempel göras med pq-formeln:
$$ms^2+cs+k=0$$
$$\Longleftrightarrow s=-\frac{c}{2m}\pm \sqrt{\bigg(\frac{c}{2m}\bigg)^2-\frac{k}{m}}$$

Här uppstår tre potentiella fall för poler beroende på uttrycket i kvadratroten, den så kallade diskriminanten.

\begin{itemize}
    \item Om diskriminanten är positiv bildas två skilda reella poler. Kravet är då: 
    $$ \bigg(\frac{c}{2m}\bigg)^2-\frac{k}{m} > 0 \,\,\, \Longleftrightarrow\,\,\, \frac{c^2}{4m} > k $$   
    \item Om diskriminanten är noll bildas en reell dubbelpol. Kravet för dubbelpolen är: 
    $$ \bigg(\frac{c}{2m}\bigg)^2-\frac{k}{m} = 0 \,\,\, \Longleftrightarrow\,\,\, \frac{c^2}{4m} = k $$
    \item Om diskriminanten är negativ bildas två komplexkonjugerade poler. Kravet för komplexa poler är: 
    $$ \bigg(\frac{c}{2m}\bigg)^2-\frac{k}{m} < 0 \,\,\, \Longleftrightarrow\,\,\, \frac{c^2}{4m} < k $$
\end{itemize}

Som nämnts innan behöver inte Laplacetransformen vara definierad för alla $s$. Var transformen konvergerar i det komplexa planet bestäms av den reella delen av $s$ och kallas konvergensområdet. För kausala system får man högersidiga konvergensområden i det komplexa planet. Då bestäms konvergensområdets vänstra gräns av den pol som är längst till höger. För vårt system får vi två fall för detta beroende på var polerna är placerade.
\begin{itemize}
    \item Om vi har en reell dubbelpol eller komplexkonjugerade poler är konvergensområdet:
    $$ Re\{s\} > -\frac{c}{2m} $$
    \item Om vi har två skilda reella poler blir konvergensområdet:
    $$ Re\{s\} > -\frac{c}{2m}+\sqrt{\bigg(\frac{c}{2m}\bigg)^2-\frac{k}{m}} $$
\end{itemize}

För att få en bättre förståelse av systemfunktionen kommer poler, nollställen och konvergensområdet visas grafiskt i ett pol-nollställediagram. För detta behöver även nivånkonstanten $K$ bestämmas. Denna fås genom att bryta ut koefficienterna för den högsta graden av $s$ i täljar- och nämnarpolynomet. I vårt fall är denna alltid:
$$K=\frac{1}{m}$$
Nedan visas pol-nollställediagramen för de tre olika fallen för polerna. Enligt standardnotation representerar kryssen poler och cirklar nollställen, dock uppstod inga nollställen. Det rödsträckade området är konvergensområdet.

\begin{figure}[H] 
    \centering
    \includegraphics[scale=0.33]{bilder/pol_nollstallediagram_2_poler}
    \caption{Pol-nollställediagram för två skilda reella poler}
    \label{fig:pol_nollstallediagram_2_poler}
\end{figure}
\begin{figure}[H] 
    \centering
    \includegraphics[scale=0.33]{bilder/pol_nollstallediagram_dubbelpol}
    \caption{Pol-nollställediagram för en reell dubbelpol}
    \label{fig:pol_nollstallediagram_dubbelpol}
\end{figure}
\begin{figure}[H] 
    \centering
    \includegraphics[scale=0.33]{bilder/pol_nollstallediagram_komplexa_poler}
    \caption{Pol-nollställediagram för två komplexkonjugerade poler}
    \label{fig:pol_nollstallediagram_komplexa_poler}
\end{figure}
 
\newpage
\subsection{Impulssvar}
Vi ska nu se hur vårt system reagerar på olika typer av insignaler. En vanligt signal som studeras är enhetsimpulsen, också kallad Dirac-pulsen, $\delta(t)$. 
Enhetsimpulsen är noll för alla $t\ne 0$. Vid $t = 0$ är den oändligt stor så att dess area är lika med $1$. Det är svårt att föreställa hur detta skulle representera sig fysikalisk i vår modell. Ett ungefärligt exempel skulle vara då man släpper en tennisboll från en hög höjd som träffar personen på huvudet och studsar bort. 

Impulssvaret $h(t)$ är utsignalen då ett system tar emot en enhetsimpuls som insignal. Denna kan räknas ut genom att inverstransformera systemfunktionen $H(s)$. Vi kommer återigen att använda tabeller för dessa beräkningar då integralerna är jobbiga att räkna ut. Man får istället problemet att skriva om uttrycken så att de matchar något i tabellen. Då det finns tre uppsättningar av poler kommer impulssvaret för dessa räknas ut separat.

\begin{itemize}
    \item Vid två skilda reella poler kan vi faktorisera rötterna i nämnarpolynomet och partialbråksuppdela för att hitta en lämplig inverstransform.
    
    $$H(s)=\frac{1}{m} \cdot \frac{1}{s^2+\frac{cs}{m}+\frac{k}{m}}=\Bigg/ \,\alpha=\frac{c}{2m}\,\,,\,\,\, \omega_0=\sqrt{\bigg(\frac{c}{2m}\bigg)^2-\frac{k}{m}} \,\,\Bigg/$$
    $$=\frac{1}{m} \cdot \frac{1}{\big(s+\alpha-\omega_0\big)\big(s+\alpha+\omega_0\big)}  = \, \frac{1}{2m\omega_0} \Bigg(\frac{1}{s+\alpha-\omega_0}-\frac{1}{s+\alpha+\omega_0}\Bigg)$$
    \begin{center}$ \Longleftrightarrow \bigg/$ Tabell $19.12\,\bigg/$ 
    $\Longleftrightarrow h(t)=\dfrac{1}{2m\omega_0}\bigg(e^{-(\alpha-\omega_0)t}-e^{-(\alpha+\omega_0)t}\bigg)u(t)$ \end{center}
    
    \item Vid reell dubbelpol kvadratkompletteras nämnarpolynomet och utnyttjas att
    $\dfrac{k}{m}-\dfrac{c^2}{4m^2}=0$.
    $$ H(s)= \frac{1}{m} \cdot\frac{1}{\big(s+\frac{c}{2m}\big)^2+\big(\frac{k}{m}-\frac{c^2}{4m^2}\big)} = \frac{1}{m} \cdot \frac{1}{\big(s+\frac{c}{2m}\big)^2}$$
    \begin{center}
    $ \Longleftrightarrow \bigg/ \alpha=\dfrac{c}{2m}\,,\,$ Tabell $19.15\,\bigg/ \Longleftrightarrow h(t)=\dfrac{1}{m} \cdot te^{-\alpha t}\,u(t)$
    \end{center}
    
    \item För komplexkonjugerade poler kvadratkompletteras nämnarpolynomet precis som innan. Sedan anpassas uttrycket med hjälp av förlängning för att överensstämma med tabellen.
    $$H(s)=\frac{1}{m} \cdot \frac{1}{\big(s+\frac{c}{2m}\big)^2+\big(\frac{k}{m}-\frac{c^2}{4m^2}\big)} = \Bigg/\, \,\alpha=\frac{c}{2m}\,\,,\,\,\,\omega_0=\sqrt{\frac{k}{m}-\frac{c^2}{4m^2}} \,\,\,\Bigg/ $$
    \begin{center}
    $=\dfrac{1}{m\omega_0} \cdot \dfrac{\omega_0}{(s+\alpha)^2+\omega_0^2} \,\,\Longleftrightarrow \bigg/$ Tabell $19.23\,\bigg/\Longleftrightarrow\,\, h(t)=\dfrac{1}{m\omega_0} \cdot e^{-\alpha t} \sin(\omega_0 t)u(t)$
    \end{center}
\end{itemize}

\subsubsection{Val av systemparametrar}
För att kunna visualisera och analysera impulssvaren måste först systemparametrarna bestämmas. Dessa bestämmer hur systemen beter sig och är i vårt fall de tre hittills obestämda konstanterna massan  $m$, fjäderkonstanten $k$ och dämpningskonstanten $c$. Vi bestämmer en standarduppsättning för systemparametrarna. För att se hur system med olika parameteruppsättningar skiljer sig åt kommer en parameter i taget varieras medan de andra två hålls konstanta.

Valet av systemparametrar baseras på scenariot som beskrivs i inledningen. Den första parametern som kommer bestämmas är massan. Denna bestäms vara $75$ kg,  vilket är den ungefärliga medelvikten hos svenska män och kvinnor enligt Statistiska centralbyrån.
Nästa parameter är fjäderkonstanten som härleds från differentialekvationen i viloläget, där konstanten $k$ kan skrivas som:
\newline$$k=\frac{mg}{y_0}$$\newline
Vi måste därför bestämma en rimlig utdragning av linan i viloläget för att sedan räkna ut $k$. Vi anser att en rimligt utdragning av linan är $0.5$ m efter att studerat bilder och filmklipp av riktiga lindansare. Vi antar att jordens tyngdacceleration $g$ kan approximeras till $9.82$ $m/s^2$. Då beräknas fjäderkonstanten till:
\newline$$k=\frac{75 \cdot 9.82}{0.5}=1473 \text{ N/m}$$

Sist har vi dämpningskonstanten $c$ som är svår att bestämma. Denna kan bestämmas till exempel experimentellt genom att mäta hur mycket systemet dämpar en insignal. Detta är inte möjligt i vårt fall och vi behöver därför resonera oss fram till ett rimligt värde. Dämpningskonstanten kommer bestämma typen av system alltså vilken typ av poler som bildas. Vi vill att systemet dämpar insignaler så lindansaren har lättare att balansera sig på linan. Man vill fortfarande ha lite gung i linan men det ska vara väldigt begränsat. Vi får denna effekt då vi har två skilda reella poler och kravet för det var:
$$\frac{c^2}{4m} > k $$
Löser vi ut $c$ och stoppar in värdena från innan får vi:
$$c>\sqrt{4km}=\sqrt{4\cdot 1473\cdot 75} \approx 665$$
Vi väljer att $c=700$ kg/s för att skapa trevliga värden när vi utför beräkningar på systemet. Standardparameteruppsättningen sammanställs då som:
\begin{center}
$\begin{cases}
\begin{aligned}
m&=75 \text{ kg} \\
k&=1473 \text{ N/m} \\
c&=700 \text{ kg/s}
\end{aligned}
\end{cases}$
\end{center}

\newpage
\subsubsection{Variation av impulssvaret}
Vi ska nu undersöka impulssvaret grafiskt för de parametrar vi har valt och vad som händer om vi ändrar på dessa. Först är vi intresserad av hur impulssvaret beter sig med vår standardparameteruppsättning alltså att $m = 75$ kg, $k=1473$ N/m och $c=700$ kg/s. Detta visas i figuren nedan.

\begin{figure}[H]
    \centering
    \includegraphics[scale=0.9]{bilder/impulssvar}
    \caption{Impulssvar med standardparameteruppsättning}
    \label{fig:impulssvar}
\end{figure}
Vi ser att impulssvaret snabbt avtar och efter cirka två sekunder är linan ungefär tillbaka till den position där den startade. Den oscillerar inte kring viloläget vilket var den effekt vi ville åt.

\newpage
Följande figur illustrerar vad som händer då massan $m$ varierar. För att lätt kunna se vad som skiljer sig åt jämförs standardmassan med massorna $20$ kg och $200$ kg. I verkligheten skulle detta vara att personer med olika vikt står på linan.
\begin{figure}[H]
    \centering
    \includegraphics[scale=0.9]{bilder/impulssvar_variation_m}
    \caption{Impulssvar med varierande massa}
    \label{fig:impulssvar_variation_m}
\end{figure}
Vi ser att lättare personer når en högre amplitud och gör det snabbare än tyngre personer. Detta är rimligt eftersom tröghet gör att tyngre personer accelereras mindre av en lika stor kraft än lättare personer. Däremot innebär också trögheten att tyngre personer kräver mer motstånd för att stanna. 

\newpage
För att få rimliga värden på $k$ betraktas linans utdragslängd $y_0$ från mittpunkten i olika lägen. Vi studerar utdragslängerna $2$ m och $0.25$ m. Detta ger fjäderkonstanen $370$ N/m respektive $3000$ N/m. I verkligheten kan man få liknande effekt genom att spänna linan olika hårt eller flytta ändpunkterna närmre eller längre bort från varandra. Figuren nedan visar impulssvaret för de olika fjäderkonstanterna.
\begin{figure}[H] 
    \centering
    \includegraphics[scale=0.9]{bilder/impulssvar_variation_k}
    \caption{Impulssvar med varierande fjäderkonstant}
    \label{fig:impulssvar_variation_k}
\end{figure}
Här ser vi tydligt att en lägre fjäderkonstant gör att massan tar mycket längre tid att ta sig tillbaka till viloläget. Tänker man det som hur hårt linan är spänd känns det rimligt då en hårdare spänd lina snabbare tar sig tillbaka till utgångsläget. 

\newpage
Nedan undersöks vad som händer då dämpningskonstanten $c$ ändras. I verkligheten skulle liknande effekt fås om man ändrar linans materialet, till exempel från ett rep till en vajer.
I figuren nedan visas impulssvaret då dämpningskonstanten halveras samt fördubblas.
\begin{figure}[H]
    \centering
    \includegraphics[scale=0.9]{bilder/impulssvar_variation_c}
    \caption{Impulssvar med varierande dämpningskonstant}
    \label{fig:impulssvar_variation_c}
\end{figure}
Vid lägre dämpningskonstant så dämpar systemet mindre och vi får en högre amplitud. Vid låg dämpning så går den också över viloläget då den inte har hunnit dämpa insignalen tillräckligt. I de andra två fallen minskar den utan oscillation och med en lägre toppamplitud desto högre dämpningskonstanten är.

\newpage
\subsection{Stegsvar}
En annan insignal vi ska studera är enhetssteget $u(t)$. Detta är en plötslig förändring av insignalen från $0$ till $1$ enligt:
$$u(t)=\begin{cases} 1, & \text{om } t \ge 0 \\ 0, & \text{om } t < 0\end{cases}$$
När ett system matas med ett enhetssteg får man stegsvaret $g(t)$ som utsignal. Detta kan i vårt system representeras av att en vikt kastas till personen på linan. Eftersom vårt system är linjärt kan denna utsignal beräknas genom att falta enhetssteget med impulssvaret som har beräknats innan. Denna uträkning kan sedan förenklas till en integral över bara impulssvaret enligt:
$$g(t)=(u*h)(t)=\int\limits_{-\infty}^{\infty}h(\tau)\,u(t-\tau)\,d\tau=\int\limits_{-\infty}^{t}h(\tau)\,d\tau$$
Våra tre typer av generella system ger upphov till tre olika impulssvar och kommer därför även ge upphov till tre olika stegsvar. För att förenkla beräkningen av dessa tre stegsvar behåller vi variabelbytena som valdes då impulssvaren beräknades. Eftersom alla impulssvar har en faktor av enhetsteget ges inget bidrag till faltningsintegralen då $t < 0$. Nedan beräknas vad som händer då $t \ge 0$.
\begin{itemize}
    \item Vid två skilda reella poler kan vi direkt integrera exponentialfunktionerna.
    $$\begin{aligned}
    g_1(t)
    &=\int\limits_{-\infty}^{t}\dfrac{1}{2m\omega_0}\bigg(e^{-(\alpha-\omega_0)\tau}-e^{-(\alpha+\omega_0)\tau}\bigg)u(\tau)\,d\tau
    \\&=\dfrac{1}{2m\omega_0}\int\limits_{0}^{t}\bigg(e^{-(\alpha-\omega_0)\tau}-e^{-(\alpha+\omega_0)\tau}\bigg)\,d\tau
    \\&=\dfrac{1}{2m\omega_0}\bigg[\frac{e^{-(\alpha+\omega_0)\tau}}{(\alpha+\omega_0)}-\frac{e^{-(\alpha-\omega_0)\tau}}{(\alpha-\omega_0)} \bigg]_{\tau=0}^{\tau=t}
    \\&=\dfrac{1}{2m\omega_0(\alpha^2-\omega_0^2)}\bigg((\alpha-\omega_0)e^{-(\alpha+\omega_0)t}-(\alpha+\omega_0)e^{-(\alpha-\omega_0)t}+2\omega_0\bigg)\end{aligned}$$
    \item Vid dubbelpol används en partiell integration för att lösa integralen.
    $$\begin{aligned}g_1(t)&=\dfrac{1}{m}\int\limits_{0}^{t}\tau e^{-\alpha \tau}\,d\tau
    \\&=\dfrac{1}{m} \bigg[\dfrac{\tau e^{-\alpha \tau}}{-\alpha}\bigg]_{\tau=0}^{\tau=t}+\dfrac{1}{m\alpha}\int\limits_{0}^{t}e^{-\alpha \tau}\,d\tau
    \\&=-\dfrac{1}{m\alpha^2}\alpha te^{-\alpha t}+\dfrac{1}{m\alpha}\bigg[\dfrac{e^{-\alpha \tau}}{-\alpha}\bigg]_{\tau=0}^{\tau=t}
    \\&=\dfrac{1}{m\alpha^2}\bigg(1-e^{-\alpha t}(\alpha t+1)\bigg)\text{\hspace{6cm}\,}\end{aligned}$$
    \newpage
    \item Vid komplexkonjugerade poler används upprepad partiell integration tills man får tillbaka uttrycket man hade från början och kan på så sätt lösa ut stegsvaret.
    $$\begin{aligned}
    g_1(t)&=\dfrac{1}{m\omega_0}\int\limits_{0}^{t} e^{-\alpha \tau} \sin(\omega_0 \tau)\,d\tau
    \\&=\dfrac{1}{m\omega_0}\bigg[\dfrac{e^{-\alpha \tau}\sin(\omega_0 \tau)}{-\alpha}\bigg]_{\tau=0}^{\tau=t}+\dfrac{1}{m\alpha}\int\limits_{0}^{t} e^{-\alpha \tau} \cos(\omega_0\tau)\,d\tau
    \\&=-\dfrac{e^{-\alpha t}\sin(\omega_0 t)}{m\omega_0\alpha}+\dfrac{1}{m\alpha}\bigg[\dfrac{e^{-\alpha \tau}\cos(\omega_0 \tau)}{-\alpha}\bigg]_{\tau=0}^{\tau=t}-\dfrac{\omega_0}{m\alpha^2}\int\limits_{0}^{t}e^{-\alpha \tau} \sin(\omega_0 \tau)\,d\tau
    \\&=-\dfrac{e^{-\alpha t}\sin(\omega_0 t)}{m\omega_0\alpha}+\dfrac{1-e^{-\alpha t}\cos(\omega_0 t)}{m\alpha^2}-\dfrac{\omega_0^2}{\alpha^2}\,g_1(t)
    \\\\\Longleftrightarrow g_1(t)
    &=\dfrac{\omega_0-e^{-\alpha t}(\alpha\sin(\omega_0 t)+\omega_0\cos(\omega_0 t))}{m\omega_0(\alpha^2+\omega^2)}\end{aligned}$$
\end{itemize}
Eftersom dessa tre stegsvar bara gäller då $t\ge 0$ kommer de multipliceras med enhetssteget för ett fullständigt stegsvar enligt:
$$g(t)=g_1(t)\cdot u(t)$$
Detta ger att stegsvaren för reella poler, dubbelpol respektive komplexa polerna blir:

$\begin{cases}
g(t)=\dfrac{(\alpha-\omega_0)e^{-(\alpha+\omega_0)t}-(\alpha+\omega_0)e^{-(\alpha-\omega_0)t}+2\omega_0}{2m\omega_0(\alpha^2-\omega_0^2)}\,u(t) \\\\
g(t)=\dfrac{1-e^{-\alpha t}(\alpha t+1)}{m\alpha^2}\,u(t) \\\\
g(t)=\dfrac{\omega_0-e^{-\alpha t}(\alpha\sin(\omega_0 t)+\omega_0\cos(\omega_0 t))}{m\omega_0(\alpha^2+\omega_0^2)}\,u(t)
\end{cases}$


\newpage
\subsubsection{Variation av stegsvaret}
Vi vill nu se hur stegsvaret beter sig för olika parameteruppsättningar. För att se hur impulssvaret och stegsvaret relaterar till varandra kommer vi använda samma parametrar som vi gjorde för variationen av impulssvaret. Återigen betraktar vi först vad som händer med vårt system med standardparameteruppsättningen, alltså då $m = 75$ kg, $k=1473$ N/m och $c=700$ kg/s. Detta görs nedan i figuren.
\begin{figure}[H]
    \centering
    \includegraphics[scale=0.9]{bilder/stegsvar}
    \caption{Stegsvar med standardparameteruppsättningen}
    \label{fig:stegsvar}
\end{figure}
Vi ser att efter cirka 1.5 sekunder har den nått ett nytt läge längre ned. Detta görs utan oscillation kring de nya viloläget. Detta skulle i verkligheten vara som  att kasta en vikt på ungefär $0.1$ kg till lindansaren.

\newpage
Nu vill vi kolla vad som händer då massan förändras. Återigen används massorna $20$ kg och $200$ kg som representerar en lättare och tyngre person på linan. I figuren nedan ser vi hur stegsvaret beter sig då massan ändras.
\begin{figure}[H]
    \centering
    \includegraphics[scale=0.9]{bilder/stegsvar_variation_m}
    \caption{Stegsvar med varierande massa}
    \label{fig:stegsvar_variation_m}
\end{figure}
Här syns inte så stor skillnad på de tre olika massorna, de närmar sig det nya höjdläget vid ungefär samma tidpunkt. Den tyngre massan tar längre tid på sig att börja röra sig men kommer även längre ned än de andra två samt skapar oscillation.
Vi kan också notera att massan inte påverkar nivåkonstanten då dessa går mot samma jämviktsläge.

\newpage
Härnäst ska fjäderkonstanen variera mellan $370$ N/m, $1473$ N/m och $3000$ N/m, vilket gäller då linan är utdragen $2$ m, $0.5$ m respektive $0.25$ m från mittpunkten. Figuren nedan visar hur stegsvaret beter sig vid dessa värden.
\begin{figure}[H]
    \centering
    \includegraphics[scale=0.9]{bilder/stegsvar_variation_k}
    \caption{Stegsvar med varierande fjäderkonstant}
    \label{fig:stegsvar_variation_k}
\end{figure}
Vi ser direkt att fjäderkonstanten påverkar det läge som personen svänger mot. Som för impulssvaret innebär en lägre fjäderkonstant att det tar längre tid att komma till ett nytt jämviktsläge. Detta är rimligt då man kan se på fjäderkonstanten som att den beskriver hur hårt linan är spänd.

\newpage
Sist varierar vi dämpningskonstanten $c$ med att både halvera och dubblera den. Denna graf visas nedan.
\begin{figure}[H]
    \centering
    \includegraphics[scale=0.9]{bilder/stegsvar_variation_c}
    \caption{Stegsvar med varierande dämpningskonstant}
    \label{fig:stegsvar_variation_c}
\end{figure}
Likt impulssvaret leder en hög dämpningskonstant till en kraftigare dämpning. Intressant nog kommer stegsvaret för de lägre dämpningskonstanerna snabbare till det nya läget än för den höga konstanten. 

\newpage
\subsection{Stabilitet}
För att bestämma systemets frekvensfunktion måste impulssvaret vara fouriertransformerbart vilket kräver att systemet är stabilt. Ett stabilt system är ett system där begränsade insignaler alltid ger begränsade utsignaler. För att bestämma om ett system är stabilt eller inte finns olika tillvägagångssätt. Ett sätt att bestämma detta är att titta på impulssvaret. Om impulssvaret är absolutintegrerbart så är systemet stabilt. Att impulssvaret är absolutintegrerbart innebär att:
$$\int\limits_{-\infty}^{\infty}\big|h(t)\big|\,dt < \infty$$
Ett annat sätt att bestämma om systemet är stabilt är att kolla på konvergensområdet för systemfunktionen. Om den imaginära axeln, $j\omega$-axeln, är i konvergensområdet så är systemet stabilt.
Detta gäller alltid för vårt system vilket kan ses i pol-nollställediagramen för systemfunktionen i kapitel $2.2.1$. Egentligen krävs också att $m,k,c>0$ men detta har antagits för att modellen ens ska vara meningsfull.
\subsection{Frekvensfunktion}
För att se hur vårt system påverkar signaler av olika frekvenser kommer vi betrakta systemets frekvenskfunktion $H(\omega)$.
Frekvensfunktionen är Fouriertransformen av impulssvaret $h(t)$. För att Fouriertransformen ska existera krävs att systemet är stabilt vilket visades i föregående kapitel. Vidare är Fouriertransformen ett specialfall av Laplacetransformen där variabeln $s = j\omega$. Frekvensfunktionen har alltså bara den reella vinkelfrekvensen $\omega$ som argument. Med insättning av $s = j\omega$ i systemfunktionen $H(s)$ fås: 

$$H(\omega)=H(s)\bigg\rvert_{s=j\omega}=\dfrac{1}{m(j\omega)^2+cj\omega+k}$$

\newpage
\subsection{Amplitudkaraktäristik}
Vi kommer nu undersöka hur systemet påverkar utsignalens amplitud beroende på frekvensens insignalen. Amplitudkaraktäristiken är absolutbelopet av frekvensfunktionen och visar hur systemet dämpar och förstärker vissa signaler. 
$$\big|H(\omega)\big|=\dfrac{1}{\Big|\,m(j\omega)^2+cj\omega+k\,\Big|}=\dfrac{1}{\sqrt{\big(k-m\,\omega^2\big)^2+\big(c\,\omega\big)^2}}$$
Det allra första som kommer undersökas är vad som händer med amplituderna för vår standardparameteruppsättning, alltså då $m=75$ kg, $k=1473$ N/m och $c=700$ kg/s. Detta visas i figuren nedan.
\begin{figure}[H]
    \centering
    \includegraphics[scale=0.9]{bilder/amplitudkaraktaristik}
    \caption{Amplitudkaraktäristik med standardparameteruppsättning}
    \label{fig:amplitudkaraktaristik}
\end{figure}
Vi ser att systemet släpper igenom låga frekvenser och dämpar höga frekvenser. Detta system beter sig som ett lågpassfilter.
Då vi har olika enheter på insignal och utsignal, en kraft respektive en längd, är det svårt att tolka hur mycket som faktiskt förstärks eller dämpas vid en viss vinkelfrekvens. Vi kan utifrån grafen se att gränsvinkelfrekvensen är ungefär $2.5$ rad/s, det vill säga då:
$$\big|H(\omega)\big|=\dfrac{max\Big\{\big|H(\omega)\big|\Big\}}{\sqrt{2}}\approx \frac{7}{\sqrt{2}} \approx 5$$
\newpage
\subsubsection{Variation av amplitudkaraktäristiken}
Vi ska nu undersöka hur amplitudkaraktäristiken beter sig med olika parameteruppsättningar. Först kollar vi vad som händer då massan ändras, vilket görs i figuren nedan.
\begin{figure}[H]
    \centering
    \includegraphics[scale=0.9]{bilder/amplitudkaraktaristik_variation_m}
    \caption{Amplitudkaraktäristik med varierande massa}
    \label{fig:amplitudkaraktaristik_variation_m}
\end{figure}
Vi ser att en tyngre massa ger en skarpare kurva. Vill man ha högre amplituder vid låga frekvenser ska man ha en stor vikt. Detta verkar rimligt då en större person är svårare att accelerera men svänger mer då den får upp fart. Betraktar vi pol-nollställediagrammet för systemen ger $m = 200$ komplexkonjugerade poler, medan $m = 20$ och $m = 75$ båda har två reellvärda poler. För en reell dubbelpol och för två reellvärda poler är amplitudkaraktäristikens topp vid $\omega = 0$, medan komplexkonjugerade poler innebär att toppen hamnar vid en nollskild vinkelfrekvens.

\newpage
I figuren nedan betraktas vad som händer med amplitudkaraktäristiken då fjäderkonstanten ändras mellan låga och högra värden. 
\begin{figure}[H]
    \centering
    \includegraphics[scale=0.9]{bilder/amplitudkaraktaristik_variation_k}
    \caption{Amplitudkaraktäristik med varierande fjäderkonstant}
    \label{fig:amplitudkaraktaristik_variation_k}
\end{figure}
Här ser vi hur fjäderkonstanten har en direkt koppling till amplituden vid låga frekvenser. Ju lösare linan är spänd desto högre vill linan svänga. Detta jämnas ut vid cirka $5$ rad/s då fjäderkonstantens påverkan drastiskt minskar. Detta är på grund av att lägesförändringen blir mycket liten vid hög frekvens och fjäderkraften motverkar läget.

\newpage
Härnäst kommer vi se hur amplitudkaraktäristiken beter sig när dämpningskonstanten ändras. Igen väljer vi att halvera och dubblera konstanten vilket visas i figuren nedan.  
\begin{figure}[H]
    \centering
    \includegraphics[scale=0.9]{bilder/amplitudkaraktaristik_variation_c}
    \caption{Amplitudkaraktäristik med varierande dämpningskonstant}
    \label{fig:amplitudkaraktaristik_variation_c}
\end{figure}
Vi ser generellt att en högre dämpningskonstant dämpar mer för alla frekvenser. Vid en lägre dämpningskonstant bildas komplexkonjugerade poler i frekvensdomänen vilket betyder att toppen av amplitudkaraktäristiken flyttas åt höger. I det fallet får vi en nollskild resonansfrekvens, vilket i grafen kan avläsas vara ungefär $3$ rad/s. 

\newpage
\subsection{Faskaraktäristik}
Faskaraktäristiken beskriver hur mycket systemet förskjuter en signal i fas beroende på frekvensen. Denna kan tas fram genom att ta argumentet för frekvensfunktionen enligt:
\\$$\begin{aligned}arg\big\{H(w)\big\}=\arg\big\{1\big\}-\arg\big\{k-m\omega^2+jc\,\omega\big\}
=\begin{cases}
-\arctan\bigg(\dfrac{c\,\omega}{k-m\omega^2}\bigg), & \text{om } k-m\omega^2 > 0  \\\\
-\arctan\bigg(\dfrac{c\,\omega}{k-m\omega^2}\bigg)-\pi, & \text{om } k-m\omega^2 < 0 \\\\ 
-\dfrac{\pi}{2}, & \text{om } k-m\omega^2 = 0 
\end{cases}\end{aligned}$$
\\I grafen nedan visas faskaraktäristiken för vår standardparameteruppsättning.
\begin{figure}[H]
    \centering
    \includegraphics[scale=0.75]{bilder/faskaraktaristik}
    \caption{Faskaraktäristik med standardparameteruppsättning}
    \label{fig:faskaraktaristik}
\end{figure}
Vi ser att vårt system fastförskjuter signaler med högre frekvenser mer än lägre. Man kan alltså säga att systemet reagerar bättre på långsamma förändringar än snabba.
Argumentet kommer i detta fall gå mot $-\pi$ då frekvensen går mot oändligheten. Detta kan bevisas då det för stora $\omega$  gäller att:
$$\lim_{\omega\to\infty}-\arctan\bigg(\dfrac{c\,\omega}{k-m\omega^2}\bigg)-\pi=
\lim_{\omega\to\infty}-\arctan\Bigg(\dfrac{1}{\omega}\cdot\frac{c}{\frac{k}{\omega^2}-m}\Bigg)-\pi=-\arctan(0)-\pi=-\pi$$
Då faskaratäristiken inte är intressant för vårt system kommer vi inte analysera den något ytterligare.
\newpage
\subsection{Stationära sinussignaler}
När ett LTI-system matas med en godtycklig sinusfunktion som insignal får vi en utsignal $y(t)$ som kan beräknas med hjälp av frekvensfunktionen $H(\omega)$. Om vi till exempel matar systemet med insignalen:
$$x(t)=A\sin(\omega_0t)$$
får vi en utsignal som kan beräknas med hjälp av frekvensfunktionen enligt:
$$y(t)=A\,\big|H(\omega_0)\big|\sin\big(\omega_0t+\arg\big\{H(\omega_0)\big\}\big)$$
Vi ser att utsignalen är en amplitudskalad och fastförskjuten version av insignalen.
Detta gör det enkelt att beräkna utsignalen om insignalen är en sinusfunktion och frekvensfunktion är bestämd. Vi ska nu se hur vårt system reagerar på olika typer av sinusfunktioner med olika frekvenser.

\subsubsection{Gångtakt}
Den första frekvensen som kommer analyseras är en som liknar takten en riktig lindansare skulle gå i. Detta för att få en ungefärlig känsla över hur systemet svänger i ett verkligt scenario. Kraften en person påverkar linan med då denne går i jämn takt kan modelleras som en sinusvåg. När personen böjer sitt ben minskas tillfälligt kraften personer påverkar linan med och när personen sträcker ut sitt ben ökar kraften tillfälligt.

Kraften personen påverkar linan med och vinkelfrekvensen måste approximeras. Kraften kan approximeras genom experiment då en person står på en våg och upprepat böjer på sina ben. Det visar sig att denna kraft är ungefär $200$ N. Vinkelfrekvensen kan också approximeras via experiment. Då vi prövade att gå i normaltakt fick vi en periodtid på cirka $1$ sekund. Genom vidare observationer av videoklipp av lindansare såg vi att detta stämde bra överens med det vi kom fram till innan. Periodtiden kan användas för att bestämma vinkelfrekvensen enligt:
$$\omega=2\pi f= \dfrac{2\pi}{T} \Longleftrightarrow \omega_0 = 2\pi$$
Insignalen kan då skrivas som:
$$x(t)=200\sin(2\pi t)$$

Detta ger följande amplitudskalade och fasförskjutna utsignal:
$$\begin{aligned}y(t)&=200\,\big|H(2\pi)\big|\sin\big(2\pi t+\arg\big\{H(2\pi)\big\}\big)
\\&=\bigg/ \,\,\big|H(2\pi)\big|\approx 0.00022 \,,\,\,\, \arg\big\{H(2\pi)\big\}\approx -1.90  \,\bigg/
\\&\approx 0.044\sin(2\pi t - 1.90)\end{aligned}$$

\newpage
Här nedan ses insignalen och utsignalen då personen går på linan. Notera att den blå insignalen har enheten newton och röda utsignalen är i millimeter. 

\begin{figure}[H]
    \centering
     \includegraphics[scale=0.9]{bilder/sinusfunktion_gangtakt}
    \caption{Insignal och utsignal för gångtakten. $x(t)$ mäts i N och $y(t)$ i mm}
    \label{fig:sinusfunktion_gangtakt}
\end{figure}
Vi ser att systemet släpper igenom den lågfrekventa signalen och bildar stora svängningar i linan. Toppamplituden är ungefär $5$ cm, alltså en positiv  lägesförändring på $5$ cm då man går och bildar en maximal förändring på $10$ cm.
Detta är en ganska stor förändring och det skulle förmodligen vara svårt att balansera sig på denna lina då man går.


\newpage
\subsubsection{Löptakt}
Det är vanligt att lindansare inte bara går utan även springer fram på linan. Därför betraktar vi även hur vårt system beter sig vid löptakt. Precis som för gången kan kraften personen påverkar linan med modelleras som en sinusfunktion. Då man springer trycker man ifrån linan mer än då man går så amplituden kommer att vara något högre. En rimlig approximation på sinusfunktionens amplitud är 250 N. Dessutom kommer frekvensen att vara högre. Videoklipp där lindansare springer fram på linan visar att en vanlig takt är tre till fyra steg per sekund. Vi väljer att betrakta fallet då personen tar tre steg per sekund. Periodtiden blir då $\sfrac{1}{3}$ sekund.
$$\omega_0 = \dfrac{2\pi}{T} = 6\pi$$
Vilket ger den sinusformade insignalen:
$$x(t) = 250\sin(6\pi t)$$

Likt innan blir utsignalen amplitudskalad och fasförskjuten. Utsignalen blir:
$$\begin{aligned}
y(t)&=250\,\big|H(6\pi)\big|\sin\big(6\pi t+\arg\big\{H(6\pi)\big\}\big)
\\&=\bigg/ \,\,\big|H(6\pi)\big|\approx 0.000035 \,,\,\,\, \arg\big\{H(6\pi)\big\}\approx -2.66  \,\bigg/
\\&\approx 0.00875\sin(6\pi t - 2.66)\end{aligned}$$

\newpage

Sist betrakar vi hur insignalen och utsignalen för löptakten förhåller sig till varandra i följande figur.
\begin{figure}[H]
    \centering
     \includegraphics[scale=0.9]{bilder/sinusfunktion_loptakt}
    \caption{Insignal och utsignal för löptakten. $x(t)$ mäts i N och $y(t)$ i mm}
    \label{fig:sinusfunktion_loptakt}
\end{figure}

Vi ser att amplituden blir mycket lägre då man springer på linan än då man går trots att man påverkar linan med en något större kraft då man springer. Detta stämmer bra överens med den amplitudkaraktäristiken som vi betraktade tidigare. Det går alltså att springa på linan för att undvika stora svängningar och på så sätt få bättre balans.
I verkligheten görs detta så klart bara då man inte dör av ett eventuellt fall då man förlorar viss kontroll av att springa. Så precis som det oftast är i livet är en gyllene medelväg att föredra. 